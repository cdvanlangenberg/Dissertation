%%%%%%%% this is the Appendix A n in data generation paper
%%%%%%%% details of deriving C_m

%\documentclass[12pt, amstex, letterpaper] {report} %{article}


\usepackage[margin=1in]{geometry}
\topmargin -0.5in \textwidth 6.5in \textheight 9in
\footskip .5in
\headheight 0.3in


\usepackage{Sweave}

\DefineVerbatimEnvironment{Sinput}{Verbatim} {xleftmargin=0em,frame=single}
\DefineVerbatimEnvironment{Soutput}{Verbatim} {xleftmargin=0em,frame=single}

\usepackage{amssymb, mathrsfs, amsmath, amsfonts}
\usepackage{enumerate, comment}
\usepackage{hyperref, natbib,apalike, float} %cite
\usepackage{color, multirow, setspace, fancyhdr,graphicx}
\usepackage{undertilde}
\usepackage[bottom]{footmisc}
\usepackage{graphicx}
\usepackage{framed}
\usepackage{subcaption}
\usepackage{amsthm}

%\doublespacing
\pagestyle{empty}
\pagestyle{fancy}
\lhead{ }
%\rhead{May 2016}
\fancyfoot{ }
\rfoot{Dissertation $|$ \thepage}
\lfoot{Chris Vanlangenberg}
\date{}

\includecomment{comment}

\newtheorem{theorem}{Theorem}[section]
\newtheorem{defn}{Definition}[section]
\newtheorem{prop}{Proposition}
\newcommand{\pro}[1]{\begin{prop}{#1}\end{prop}}

%\newtheorem{proof}{proof}
\newtheorem{rmk}{Remark}
\newcommand{\rmark}[1]{\begin{rmk}{#1}\end{rmk}}

\numberwithin{equation}{section}
\renewcommand{\footrulewidth}{0.1pt}
\renewcommand{\headrulewidth}{0.1pt}


\newcommand{\eqn}[1]{\begin{equation}{#1}\end{equation}}

\newcommand{\beq}{\begin{equation}}
\newcommand{\eeq}{\end{equation}}
%\renewcommand\refname{Literature}
\newcommand{\blue}[1]{\textcolor{blue}{\emph{#1}}}
\newcommand{\red}[1]{\textcolor{red}{\emph{#1}}}
\newcommand{\twoc}[2]{{\textcolor{blue}{#1}} and {\textcolor{red}{#2}}}


\newcommand{\xn}{x_1,\ldots, x_n}
\newcommand{\Xn}{X_1,\ldots, X_n}
\newcommand\floor[1]{\lfloor{#1}\rfloor}
\newcommand\ceil[1]{\lceil{#1}\rceil}

\newcommand{\X}{\mathcal{X}}
\newcommand{\Sp}{\mathbb{S}}
\newcommand{\R}{\mathbb{R}}
\newcommand{\C}{\mathbb{C}}
\newcommand{\pd}{positive definite }



\newcommand{\code}[1]{{\small\texttt{#1}}}
\newcommand{\pkg}[1]{{\normalfont\textsf{#1}}}
\newcommand{\var}[1] {{\normalfont\textbf{#1}}}
\newcommand{\Cm}{$C_m(\phi_P, \phi_Q)\ $}

\newcommand{\jun}{\cite{JunStein2008}}
%\begin{document}

As we note that it is very crucial to derive \Cm based on the given $R(P,Q)$. Here is the detailed steps to derive \Cm from model 1 (\ref{model1}). \Cm for other models can be obtained similarly. 
\begin{eqnarray*}
	R(P, Q) &=& R(\phi_P, \phi_Q, \Delta \lambda) = \tilde{C}(\phi_P, \phi_Q) \frac{1-p^2}{1 - 2p \cos(\Theta)+p^2},
\end{eqnarray*}
where $\Theta = \Delta \lambda + u(\phi_P - \phi_Q),$ with some choice of $C_1, C_2, a, u,$ and $p$.

Now,
\begin{eqnarray*}
	C_m(\phi_P, \phi_Q) &=& \frac{1}{2\pi} \int_{-\pi}^\pi R(\phi_P, \phi_Q, \Delta \lambda) e^{-im\Delta \lambda}d\Delta \lambda \\
	&=& \tilde{C}(\phi_P, \phi_Q) \frac{1}{2\pi}\int_{-\pi}^\pi \frac{1-p^2}{1 - 2p \cos(\Theta+p^2} e^{-im\Delta \lambda}d\Delta \lambda.
\end{eqnarray*}

Next we focus on the following integration.
\begin{eqnarray*}
	\int_{-\pi}^\pi \frac{1-p^2}{1 - 2p \cos(x+b)+p^2} e^{-imx}dx,
\end{eqnarray*}
where we set $x=\Delta\lambda$ and $b=u(\phi_P -\phi_Q)$ and we have,
\begin{eqnarray*}
	\frac{1-p^2}{1 - 2p \cos(x+b)+p^2} &=& \frac{2-2p\cos(x+b)-(1-2p \cos(x+b)+p^2)}{1-2p \cos(x+b)+p^2}\\
	&=& 2\times \frac{1-p\cos(x+b)}{1-2p \cos(x+b)+p^2}-1 \\
	&=& 2\times \sum_{n=0}^{\infty}p^n\cos n(x+b)-1  \\
	&=& 1 + 2 \sum_{n=1}^{\infty}p^n (\cos nx \cos(nb) - \sin(nx) \sin(nb)).
\end{eqnarray*}
Therefore, for $m \ne 0$,
\begin{eqnarray*}
	& & \int_{-\pi}^\pi \frac{1-p^2}{1 - 2p \cos(x+b)+p^2} e^{-imx}dx \\
	&=& \int_{-\pi}^\pi \left[1 + 2 \sum_{n=1}^{\infty}p^n (\cos nx \cos(nb) - \sin(nx) \sin(nb))\right]  e^{-imx}dx \\
	&=& \int_{-\pi}^\pi e^{-imx}dx + 2 \sum_{n=1}^{\infty}p^n \int_{-\pi}^\pi \left[\cos nx \cos(nb) - \sin(nx) \sin(nb)\right]  e^{-imx}dx \\
	&=& 2 \sum_{n=1}^{\infty}p^n \left[\cos(nb)  \int_{-\pi}^\pi \cos(nx) e^{-imx} dx - \sin(nb)  \int_{-\pi}^\pi \sin(nx) e^{-imx}dx \right] \\
	&=& 2 \sum_{n=1}^{\infty}p^n \left[ \pi \cos(nb) \delta(n, m) + \pi i\sin(nb)\right] \\
	&=& 2\pi p^m e^{imb}.
\end{eqnarray*}
That is, for $m \ne 0$,
\begin{eqnarray*}
	C_m(\phi_P, \phi_Q)  &=& \tilde{C}(\phi_P, \phi_Q) \frac{1}{2\pi} (2\pi p^m e^{imb}) \\
	&=& \tilde{C}(\phi_P, \phi_Q) p^m e^{imb},
\end{eqnarray*}
and for $m = 0$, $C_0(\phi_P, \phi_Q) = \tilde{C}(\phi_P, \phi_Q)$.\\

In summary,
\begin{eqnarray*}
	C_m(\phi_P, \phi_Q) &=& \left\{ \begin{array}{ll}
	\tilde{C}(\phi_P, \phi_Q), & m = 0 \\
	\tilde{C}(\phi_P, \phi_Q)p^m e^{imb}, & m \ne 0.
	\end{array}
	\right.
	%&=& C_1\left(C_2 - e^{-a|\phi_P|} - e^{-a|\phi_Q|} + e^{-a|\phi_P - \phi_Q|}\right)p^m (\cos(mu(\phi_P - \phi_Q))+i \sin(mu(\phi_P - \phi_Q))).
\end{eqnarray*}

% If the process is longitudinally reversible one can set $u=0$. Suppose 

If $\tilde{C}(\phi_P, \phi_Q)$ is  given by \eqref{Cm_model1} then \Cm for model1 is given by,
\begin{eqnarray*}
	C_m(\phi_P, \phi_Q) &=& \left\{ \begin{array}{ll}
	C_1\left(C_2 - e^{-a|\phi_P|} - e^{-a|\phi_Q|} + e^{-a|\phi_P - \phi_Q|}\right)     & m = 0 \\
	C_1\left(C_2 - e^{-a|\phi_P|} - e^{-a|\phi_Q|} + e^{-a|\phi_P - \phi_Q|}\right)p^m  & m \ne 0.
	\end{array}
	\right.
\end{eqnarray*}




%\end{document}
