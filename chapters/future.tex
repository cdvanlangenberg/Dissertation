%\documentclass[12pt, amstex, letterpaper] {report} %{article}


\usepackage[margin=1in]{geometry}
\topmargin -0.5in \textwidth 6.5in \textheight 9in
\footskip .5in
\headheight 0.3in


\usepackage{Sweave}

\DefineVerbatimEnvironment{Sinput}{Verbatim} {xleftmargin=0em,frame=single}
\DefineVerbatimEnvironment{Soutput}{Verbatim} {xleftmargin=0em,frame=single}

\usepackage{amssymb, mathrsfs, amsmath, amsfonts}
\usepackage{enumerate, comment}
\usepackage{hyperref, natbib,apalike, float} %cite
\usepackage{color, multirow, setspace, fancyhdr,graphicx}
\usepackage{undertilde}
\usepackage[bottom]{footmisc}
\usepackage{graphicx}
\usepackage{framed}
\usepackage{subcaption}
\usepackage{amsthm}

%\doublespacing
\pagestyle{empty}
\pagestyle{fancy}
\lhead{ }
%\rhead{May 2016}
\fancyfoot{ }
\rfoot{Dissertation $|$ \thepage}
\lfoot{Chris Vanlangenberg}
\date{}

\includecomment{comment}

\newtheorem{theorem}{Theorem}[section]
\newtheorem{defn}{Definition}[section]
\newtheorem{prop}{Proposition}
\newcommand{\pro}[1]{\begin{prop}{#1}\end{prop}}

%\newtheorem{proof}{proof}
\newtheorem{rmk}{Remark}
\newcommand{\rmark}[1]{\begin{rmk}{#1}\end{rmk}}

\numberwithin{equation}{section}
\renewcommand{\footrulewidth}{0.1pt}
\renewcommand{\headrulewidth}{0.1pt}


\newcommand{\eqn}[1]{\begin{equation}{#1}\end{equation}}

\newcommand{\beq}{\begin{equation}}
\newcommand{\eeq}{\end{equation}}
%\renewcommand\refname{Literature}
\newcommand{\blue}[1]{\textcolor{blue}{\emph{#1}}}
\newcommand{\red}[1]{\textcolor{red}{\emph{#1}}}
\newcommand{\twoc}[2]{{\textcolor{blue}{#1}} and {\textcolor{red}{#2}}}


\newcommand{\xn}{x_1,\ldots, x_n}
\newcommand{\Xn}{X_1,\ldots, X_n}
\newcommand\floor[1]{\lfloor{#1}\rfloor}
\newcommand\ceil[1]{\lceil{#1}\rceil}

\newcommand{\X}{\mathcal{X}}
\newcommand{\Sp}{\mathbb{S}}
\newcommand{\R}{\mathbb{R}}
\newcommand{\C}{\mathbb{C}}
\newcommand{\pd}{positive definite }



\newcommand{\code}[1]{{\small\texttt{#1}}}
\newcommand{\pkg}[1]{{\normalfont\textsf{#1}}}
\newcommand{\var}[1] {{\normalfont\textbf{#1}}}
\newcommand{\Cm}{$C_m(\phi_P, \phi_Q)\ $}

\newcommand{\jun}{\cite{JunStein2008}}
%\begin{document}

In this dissertation, we focus on data generation and estimation for axially symmetric processes on the sphere. We first show that for the stationary random process on the circle, the commonly used covariance function estimator based on MOM is biased with non-estimable bias, while the unbiased MOM variogram estimator is inconsistent. Our second project emphasizes on data generation, in which the axially symmetric random process can be decomposed as Fourier series on circles, where the Fourier random coefficients can be expressed as circularly-symmetric complex random vectors. We develop an algorithm that generates axially symmetric data with the given covariance model. All of the above results and theories have been supplemented via simulations. \\

We can extend this dissertation work to a number of future research areas. We will first explore the unbiasedness and consistency of the MOM covariance and variogram estimators for homogeneous and axially symmetric random processes on the sphere. In particular, we expect a similar result as the one for circle holds for axially symmetric random processes. On the other hand, we will also explore the ergodic condition (if exists) that ensures the consistency of these estimators.\\

We have noticed that our proposed data generation algorithm assumes the closed form of $C_m(\phi_P, \phi_Q)$, which sometimes may not be available. This might restrict the applicability of our data generation algorithm. Note that in order to implement the algorithm, we only need the $C_m(\phi_P, \phi_Q)$ over the gridded locations. Therefore, given the covariance structure $R(P, Q)$, we may use the Discrete Fourier Transform to obtain those gridded $C_m(\phi_P, \phi_Q)$ values. This would definitely complement our dissertation research.\\

Kriging, or making predictions at unobserved locations, has always been one of the important applications of data modeling and analysis in spatial statistics. With the complexity and dimensionality of global data, it is highly demanded that practically useful parametric models with interpretable parameters would be available for geography and environmental scientists. As the continuation of this dissertation research, we wish to enhance the kriging techniques and make use of proposed global data generation methods to make global predictions with less dimensionality.

%\end{document}