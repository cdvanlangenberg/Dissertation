%%-------------------------Random process on sphere---------------------------------------%%

A random process is a collection of random variables $X \in \{X(s): s\in D\}$, defined in a common probability space. In general, if
\begin{itemize}
	\item $s \in N$: random sequence, time series
	\item $s \in R^1$: random process which is a stochastic process
	\item $s \in R^d$: random filed, spatial process if $d > 1$
	\item $s \in S^2$: random process on the sphere
	\item $s \in R^d\times R$: spatio-temporal process
\end{itemize}

In this dissertation, we will focus only on spatial processes on a sphere, where $s$ represents a location.


%%------------------------------------------------------------------%%
\section{Random process on a sphere} 
%%------------------------------------------------------------------%%


We consider a complex-valued random process $X(P)$ on a unit sphere $S^2$, where $P=(\lambda, \phi) \in S^2$ with longitude $\lambda \in [-\pi, \pi)$ and latitude $\phi \in [0, \pi]$. Assume the process is continuous in quadratic mean with respect to the location $P$, and has finite second moment, then $X(P)$ can be represented by spherical harmonics, with convergence of the series in quadratic mean, \cite{LiNorth1997}.
	\[
		X(P) = X(\lambda,\phi) = \sum_{\nu=0}^\infty \sum_{m=-\nu}^{\nu} Z_{\nu,m} e^{i m \lambda} P_{\nu}^m (\cos \phi),
	\]
	where $P_{\nu}^m(\cdot)$ is a normalized associated Legendre polynomial so that its squared integral on $[-1, 1]$ is 1, and $Z_{\nu,m}$ are the coefficients satisfying
	\[
		Z_{\nu,m} = \int_{S^2} X(P) e^{-im \lambda} P_{\nu}^m (\cos \phi) d P.
	\]
	Without loss of generality, the process is assumed to have mean zero, i.e., $\mbox{E}(X(P)) = 0$, which implies $\mbox{E} (Z_{\nu,m}) = 0$. Then, the covariance function of the process at two locations $P = (\lambda_P, \phi_P)$ and $Q=(\lambda_Q, \phi_Q)$ is given by,
	\[
		R(P, Q) = \mbox{E}(X(P) \overline{X(Q)}) = \sum_{\nu=0}^\infty \sum_{\mu=0}^\infty \sum_{m=-\nu}^{\nu} \sum_{n=-\mu}^{\mu} \mbox{E}(Z_{\nu,m} \overline{Z}_{\mu,n}) e^{im \lambda_P} P_{\nu}^m(\cos \phi_P) e^{-i n \lambda_Q} P_{\mu}^n (\cos \phi_Q).
	\]
	where $\bar{Z}$ denotes the complex conjugate of $Z$. Note that the continuity of $X(P)$ on every point $P$ implies that $R(P, Q)$ is continuous on all pairs $(P, Q)$ \cite[page 83]{Leadbetter1967}. If the covariance function depends solely on the spherical distance between these two locations, the process is homogeneous. That is, (Obukhov, 1947, \cite{Yaglomothers1961})
	\[
		R(P, Q) = R(\theta(P,Q)) = \sum_{\nu=0}^\infty \frac{(2\nu+1)f_{\nu}}{2} P_{\nu}(\cos \theta(P,Q)),
	\]
	where the spherical distance $ \theta(P,Q) = \cos^{-1} (\cos \phi_P \cos \phi_Q + \cos \phi_P \cos \phi_Q \cos (\lambda_P-\lambda_Q))$, $P_{\nu}(\cdot)$ is the Legendre polynomial of order $\nu$, $f_{\nu} \ge 0$, and $\sum_{\nu = 0}^{\infty} (2\nu+1)f_{\nu} < \infty$. Here, the random variable $Z_{\nu,m}$ satisfies
	\[
		\mbox{E} (Z_{\nu,m} \overline{Z}_{\mu,n}) = \delta_{\nu,\mu} \delta_{n,m} f_\nu,
	\]
	where $\delta_{a, b} = 1$ if $a = b$, and $0$ otherwise.
	
	
	%%------------------------------------------------------------------%%
	\section{Axially symmetry}
	%%------------------------------------------------------------------%%
	
	The idea was introduced by \cite{Jones1963}, if the covarince between two spatial points depends only on the longitudes only through their difference  between two points then process is said to be axially symmetric.
	
	Under the assumption of axial symmetry , where the covariance function depends on longitudinal differences, one has
	\[
		\mbox{E} (Z_{\nu,m} \overline{Z}_{\mu,n}) = \delta_{n,m} f_{\nu,\mu,m}.
	\]
	Hence, the covariance function is of the form
	\begin{eqnarray} \label{axially-symmetry-cov}
		R(P,Q) = R(\phi_P, \phi_Q, \lambda_P-\lambda_Q) = \sum_{m=-\infty}^{\infty} \sum_{\nu=|m|}^\infty \sum_{\mu=|m|}^\infty f_{\nu,\mu,m} e^{im (\lambda_P-\lambda_Q)} P_{\nu}^m(\cos \phi_P) P_{\mu}^m (\cos \phi_Q).
	\end{eqnarray}
	Further conditions on $f_{\nu,\mu,m}$ are imposed in order to have the covariance function valid. In particular, $f_{\nu,\mu, m} = \overline{f}_{\mu, \nu, m}$ and for each fixed integer $m$, the matrix $F_m(N) = \{ f_{\nu,\mu,m} \}_{\nu,\mu=|m|,|m|+1, \ldots, N }$ must be positive definite for all $N \ge |m|$. A detailed discussion of parallel conditions on $f_{\nu, \mu, m}$ under the real-valued case is given in \cite{Jones1963}.
