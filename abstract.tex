\begin{abstract}
\textcolor{red}{Review and update later}

\noindent Global-scale processes and phenomena are of utmost importance in the geophysical sciences. Data from global networks and satellite sensors have been used to monitor a wide array of processes and variables, such as temperature, precipitation, etc. In this dissertation, we are planning to achieve explicitly the following objectives,

\begin{enumerate}

\item Develop both non-parametric and parametric approaches to model global data dependency.

\item Generate global data based on given covariance structure.

\item Develop kriging methods for global prediction.

\item Explore one or more of the popularly discussed global data sets in literature such as MSU (Microwave Sounding Units) data, the tropospheric temperature data from National Oceanic and Atmospheric and TOMS (Total Ozone Mapping Spectrometer) data, total column ozone from the Laboratory for Atmospheres at NASA's Goddard Space Flight Center Administration satellite-based Microwave Sounding Unit.

\end{enumerate}

\noindent Global scale data have been widely studied in literature. A common assumption on describing global dependency is the second order stationarity. However, with the scale of the Earth, this assumption is in fact unrealistic. In recent years, researchers have focused on studying the so-called axially symmetric processes on the sphere, whose spatial dependency often exhibit homogeneity on each latitude, but not across the latitudes due to the geophysical nature of the Earth. In this research, we have obtained some results on the method of non-parametric estimation procedure, in particular, the method of moments, in the estimation of spatial dependency. Our initial result shows that the spatial dependency of axially symmetric processes exhibits both anti-symmetric and symmetric characteristics across latitudes. We will also discuss detailed methods on generating global data and finally we will outline our methodologies on kriging techniques to make global prediction.

\end{abstract}