%%%%%%%%%%%%%%%%%%%%%%%%%%%%%%%%%%%
% Introduction and General Definitions
%%%%%%%%%%%%%%%%%%%%%%%%%%%%%%%%%%%

%\documentclass[12pt, amstex, letterpaper] {report} %{article}


\usepackage[margin=1in]{geometry}
\topmargin -0.5in \textwidth 6.5in \textheight 9in
\footskip .5in
\headheight 0.3in


\usepackage{Sweave}

\DefineVerbatimEnvironment{Sinput}{Verbatim} {xleftmargin=0em,frame=single}
\DefineVerbatimEnvironment{Soutput}{Verbatim} {xleftmargin=0em,frame=single}

\usepackage{amssymb, mathrsfs, amsmath, amsfonts}
\usepackage{enumerate, comment}
\usepackage{hyperref, natbib,apalike, float} %cite
\usepackage{color, multirow, setspace, fancyhdr,graphicx}
\usepackage{undertilde}
\usepackage[bottom]{footmisc}
\usepackage{graphicx}
\usepackage{framed}
\usepackage{subcaption}
\usepackage{amsthm}

%\doublespacing
\pagestyle{empty}
\pagestyle{fancy}
\lhead{ }
%\rhead{May 2016}
\fancyfoot{ }
\rfoot{Dissertation $|$ \thepage}
\lfoot{Chris Vanlangenberg}
\date{}

\includecomment{comment}

\newtheorem{theorem}{Theorem}[section]
\newtheorem{defn}{Definition}[section]
\newtheorem{prop}{Proposition}
\newcommand{\pro}[1]{\begin{prop}{#1}\end{prop}}

%\newtheorem{proof}{proof}
\newtheorem{rmk}{Remark}
\newcommand{\rmark}[1]{\begin{rmk}{#1}\end{rmk}}

\numberwithin{equation}{section}
\renewcommand{\footrulewidth}{0.1pt}
\renewcommand{\headrulewidth}{0.1pt}


\newcommand{\eqn}[1]{\begin{equation}{#1}\end{equation}}

\newcommand{\beq}{\begin{equation}}
\newcommand{\eeq}{\end{equation}}
%\renewcommand\refname{Literature}
\newcommand{\blue}[1]{\textcolor{blue}{\emph{#1}}}
\newcommand{\red}[1]{\textcolor{red}{\emph{#1}}}
\newcommand{\twoc}[2]{{\textcolor{blue}{#1}} and {\textcolor{red}{#2}}}


\newcommand{\xn}{x_1,\ldots, x_n}
\newcommand{\Xn}{X_1,\ldots, X_n}
\newcommand\floor[1]{\lfloor{#1}\rfloor}
\newcommand\ceil[1]{\lceil{#1}\rceil}

\newcommand{\X}{\mathcal{X}}
\newcommand{\Sp}{\mathbb{S}}
\newcommand{\R}{\mathbb{R}}
\newcommand{\C}{\mathbb{C}}
\newcommand{\pd}{positive definite }



\newcommand{\code}[1]{{\small\texttt{#1}}}
\newcommand{\pkg}[1]{{\normalfont\textsf{#1}}}
\newcommand{\var}[1] {{\normalfont\textbf{#1}}}
\newcommand{\Cm}{$C_m(\phi_P, \phi_Q)\ $}

\newcommand{\jun}{\cite{JunStein2008}}
%
%\begin{document}
%\bibliographystyle{apalike}

\begin{description}
\item [Strict stationarity:] for all finite $n,\ \xn \in \mathbb{R}^d$, $h_1, \ldots, h_n\in\mathbb{R} \mbox{ and } x\in \mathbb{R}^d$,
\beq
P\{Z(x_1+x)\le h_1, \ldots, Z(x_n+x)\le h_n\} = P\{ Z(x_1)\le h_1,\ldots, Z(x_n)\le h_n\}
\eeq
  \item [Weak stationarity:] If a random process $Z(x)$ has a finite second moment, it's mean function is constant and the covariance is a function of distance $C(h)=Cov(Z(x),Z(x+h))$ also referred as auto-covariance).\\
Note: Strictly stationary random fields with finite second moments is also weakly stationary.

\item [Intrinsic stationarity:] If the variance between two locations depend only on the distance of two locations. Weak stationarity implies intrinsic stationarity.

\item [Isotrophy:] The invariance property of stationarity due to rotations and reflections. If $Z(x)$ is a random field on $\mathbb{R}^d$ it is strictly isotropic if the joint distributions are invariant under all rigid motions. {\em i.e.,} for any orthogonal $d\times d$ matrix $H$ and any $x\in \R^d$
\beq
P\{Z(Hx_1+x)\le h_1, \ldots, Z(Hx_n+x)\le h_n\} = P\{ Z(x_1)\le h_1,\ldots, Z(x_n)\le h_n\}
\eeq
Isotropy assumes that it is not required to distinguish one direction from another for the random field $Z(x)$.

\item [Autocovariance:] Suppose $Z(x)$ is weakly stationary on $\R^d$ with autocavariance function $C(h) = cov(Z(x), Z(x+h))$ then,
\begin{eqnarray*}
C(0)& \ge & 0 \\
C(h)&=& C(-h)\\
|C(h)|& \le & C(0)\\
\end{eqnarray*}
Note: $C(\cdot)$ function is p.d. if
$$\sum_{i,j=1}^{N} a_i a_j C(x_i - x_j) \ge 0,$$
for any integer $N$, any constants $a_1, a_2, \ldots, a_N$, and any locations $x_1, x_2, \ldots, x_N$. \\

\textbf{Some properties of a p.d. function}
\begin{enumerate}
\item If $C_1, C_2$ are p.d. then $a_1C_1+a_2C_2$ is p.d. $\forall a_1,a_2\ge 0$
\item If $C_1, C_2, \ldots$ are p.d. and $\underset{n\to\infty} {lim}\ C_n(x)=C(x),\ \forall x\in \R^d$, then $C$ is p.d.
\item If $C_1, C_2$ are p.d. then $C(x)=C_1(x)C_2(x)$ is p.d.
\end{enumerate}

\item [Semivariogram:] an alternative to the covariance function proposed by Matheron. For an intrinsically stationary random field $Z(s)$,
\begin{align}
E[Z(s)] &= \mu , \nonumber \\
\gamma(h) &= \frac{1}{2} Var(Z(s+h) -Z(s)),
\end{align}
Where $\gamma$ is the semivariogram and $\gamma(h) = C(0) - C(h)$ for a weakly stationary process with covariance function $C(h)$.

\item [Mean square continuity \& differentiability:] there is no simple relationship between $C(h)$ and the smoothness of $Z(x)$. For a sequence of random variables $X_1, X_2,\ldots$ and a random variable $X$ defined on a common probability space. Define,$X_n\overset{L^2}\to X$ if, $E(X^2)<\infty$ and $E(X_n - X)^2\to 0$ as $n \rightarrow \infty$. We can say, $\{X_n\}$ converges in $L^2$ if there exists such a $X$.\\

Suppose $Z(x)$ is a random field on $\R^d$, Then $Z(x)$ is mean square continuous at $x$ if, $$\lim_{h\to 0} E(Z(x+h)-Z(x))^2 =0$$
If $Z(x)$ is weak stationary and $C(\cdot)$ is the covariance function then $E(Z(x+h)-Z(x))^2=2(C(0)-C(h))$. Therefore $Z(x)$ is mean square continuous iff $C(\cdot)$ is continuous at the origin.

\item [Spectral methods:] Sometimes it is convenient to use complex valued random functions, rather than real valued random functions. \\

We say, $Z(x)=U(x) + i V(x)$ is a complex random field if $U(x),V(x)$ are real random fields. If $U(x),V(x)$ are weakly stationary so does $Z(x)$.The covariance function can be defined as,
\begin{eqnarray*}
C(h) = cov(Z(x+h), \overline{Z(x)}), \quad C(-x)=\overline{C(x)},
\end{eqnarray*}
for any complex constants $c_1,\ldots, c_n,$ and any locations $x_1, x_2, \ldots, x_n$,

\beq \sum_{i,j=1}^n c_i\bar{c_j}C(x_i-x_j)\ge 0\eeq

\item [Spectral representation of a random field:] Suppose $\omega_1,\ldots, \omega_n \in \mathbb{R}^d$ and let $Z_1, \ldots, Z_n$ be mean zero complex random variables with  $E(Z_i\bar{Z_j})=0, i\ne j\ and\ E|Z_i|^2=f_i$. Then the random sum
\beq Z(x) = \sum_{k=1}^n Z_k e^{i\omega_k^Tx}.\eeq
Then $Z(x)$ given above is a weakly stationary complex random field in $\mathbb{R}^d$ with covariance function $C(x) = \sum_{k=1}^n f_k e^{i\omega_k^Tx}$\\

Further, if we think about the integral as a limit in $L^2$ of the above random sum, then the covariance function can be represented as,
\beq C(x) = \int_{\mathbb{R}^d} e^{i\omega^Tx} F(d\omega)\eeq
where $F$ is the so-called spectral distribution. For more details one can refer to \cite{stein1999}[p. 24],  Here is a more general result from Bochner.

\begin{thm}[Bochner's Theorem]\hfill \\
A complex valued covariance function $C(\cdot)$ on $\mathbb{R}$ for a weakly stationary mean square continuous complex-valued random field on $\mathbb{R}^d$ iff it can be represented as above, where $F$ is a positive measure.
\end{thm}

If $F$ has a density with respect to Lebesgue measure (spectral density) denoted by $f$, ($i.e.$ if such $f$ exists) we can use the inversion formula to obtain $f$
\beq f(\omega) = \frac{1}{(2\pi)^d}  \int_{\mathbb{R}^d} e^{-i\omega^Tx} C(x) dx \eeq

\item [Septral densities:  Rational Functions] that are even,  nonnegative and integrable the corresponding covariance functions can be expressed in terms of elementary functions. For example if $f(\omega) =\phi (\alpha^2+\omega^2)^{-1}$, then $C(h) = \pi\phi\alpha^{-1}e^{-\alpha|h|}$ (obtained by contour integration).

\item [Septral densities: Gaussian Model]
Commonly used covariance function for a smooth process on $\mathbb{R}$ where the covariance function is given by $C(h)=ce^{-\alpha h^2}$ and the corresponding spectral density is $ f(\omega) = \frac{1}{2\sqrt{\pi\alpha}}c e^{\frac{-\omega^2}{4\alpha}}$.

\item [Septral densities: Matern class] has more practical use and more frequently used in spatial statistics. The spectral density of the form $f(\omega) =\frac{1}{\phi(\alpha^2+\omega^2)^{\nu+1/2}}$ where $\phi,\nu,\alpha>0$ and the corresponding covariance function  is
\beq \label{matern_var} C(h) = \frac{\pi^{1/2}\phi}{2^{\nu-1}\Gamma(\nu+1/2)\alpha^{2\nu}} (\alpha|h|)^{\nu} Y_{\nu}(\alpha|t|),\eeq

where $Y_{\nu}$ is the modified Bessel function, the larger the $\nu$ smoother the $Y$. Further, $Y$ will be $m$ times square differntiable iff $\ \nu>m$. When $\nu$ is in the form of $m+1/2$ with $m$ a non negative integer, the spectral density is rational and the covariance function is in the form of $e^{-\alpha|h|}\cdot$ polynomial$(|h|)$ \\
\begin{eqnarray*}
\nu = 1/2 &:& C(h) = \pi\phi\alpha^{-1}e^{-\alpha|h|}\\
\nu = 3/2 &:& C(h) = \frac{1}{2}\pi\phi\alpha^{-3}e^{-\alpha|h|}(1+\alpha|h|)\\
\end{eqnarray*}

\item [{\bf $Mat\acute{e}rn$} Variance:] We can use the covariance function given by (\ref{matern_var}) to compute the Matern variances at different latitudes. The spectral density of the rational form was fitted to residuals of TOMS data.


\end{description}





%\bibliography{biblography}
%\end{document}
